\documentclass[a4paper, 12pt]{article}
\usepackage[T1]{fontenc}
\usepackage[scale=1,angle=0,opacity=1,color=black!60]{background}
\usepackage{tikzpagenodes}
\usepackage{lastpage}
\usepackage{lmodern}
\usepackage{float}
\usepackage[textwidth=420pt,textheight=630pt]{geometry}
\setlength{\oddsidemargin}{15.5pt}
%\usepackage[none]{hyphenat} %no cortar palabras

\usepackage[spanish, activeacute]{babel} %Definir idioma español
\usepackage[utf8]{inputenc} %Codificacion utf-8
\backgroundsetup{contents={}} %Saca el 'draft'
\definecolor{mygray}{rgb}{0.95,0.95,0.95}

\usepackage{listings}
\lstset{
    basicstyle=\footnotesize,
    backgroundcolor=\color{mygray},         
    breaklines=true,
    breakatwhitespace=true,   
    postbreak=\mbox{\textcolor{red}{$\hookrightarrow$}\space},              
    captionpos=b,                    
    keepspaces=true,                 
    numbers=left,                    
    numbersep=5pt,                  
    showspaces=false,                
    showstringspaces=false,
    showtabs=false,
    tabsize=4,
    language=C,
    frame=none,
	title=\lstname,
}

\def\labelitemi{$\bullet$}

\begin{document}		
	% TÍTULO, AUTORES Y FECHA
	\begin{titlepage}
		\vspace*{\fill}
		\begin{center}
			\Large 75.08 Sistemas Operativos \\
			\Huge Entrega Kernel 2 \\
			\bigskip\bigskip\bigskip
			\large\textbf{Nombre y Apellido:} Sebastián Ezequiel Blanco \\
			\textbf{Padrón:} 98539 \\
			\large\textbf{Nombre y Apellido:} Martín Nicolás Pérez \\
			\textbf{Padrón:} 97378 \\
			\textbf{Fecha de Entrega:} 22/06/2018\\
			\textbf{GitHub:} https://github.com/BlancoSebastianEzequiel/LabKernel2\\

		\end{center}
		\vspace*{\fill}
	\end{titlepage}
	\pagenumbering{arabic}
	\newpage
			
	% ÍNDICE
	\tableofcontents
	\newpage
	%\pagenumbering{arabic}
	
	\section{Creación de stacks en el kernel}
		% EJ KERN2 STACK
		\subsection{Ej: kern2-stack}
		\begin{itemize}
			\item \textit{Explicar: ¿qué significa “estar alineado”?\\}
		Estar alineado significa que nuestro espacio de memoria será
		múltiplo de un numero que es el que se utiliza para alinear, es decir,
		que los primeros n bit empezaran en cero.\\\\
			\item \textit{Mostrar la sintaxis de C/GCC para alinear a 32 bits
			el arreglo kstack anterior.\\}
\begin{lstlisting}
	unsigned char stack1[8192] __attribute__((aligned(32)));
\end{lstlisting}
			\item \textit{A qué valor se está inicializando kstack?
			¿Varía entre la versión C y la versión ASM? (Leer la documentación
			de as sobre la directiva .space.)\\}
		Esta inicializado en cero ya que el segundo argumento de space que es
		fill no es especificado, y en tal caso su valor es cero. Si varia en la
		version c ya que en esta ultima el valor de la stack en C tiene datos
		basura\\\\
			\item \textit{Explicar la diferencia entre las directivas .align y
			.p2align de as, y mostrar cómo alinear el stack del kernel a 4 KiB
			usando cada una de ellas.\\}
		Ambas rellenean el contador de ubicación (en la subsección actual) con
		un límite de almacenamiento particular. La primera expresión
		(que debe ser absoluta) es la alineación requerida. La diferencia es que
		align alinea por el valor que se le pasa y p2align alinea por el valor
		que se le pasa como potencia de dos, es decir que align 4096 alinea a
		4Kb y p2align 12 también.\\\\
			\item \textit{Finalmente: mostrar en una sesión de GDB los valores
			de esp y eip al entrar en kmain, así como los valores almacenados
			en el stack en ese momento.\\}
\begin{lstlisting}[
	columns=fullflexible,
	basicstyle=\fontfamily{lmvtt}\selectfont,
]
	$ make gdb
	gdb -q -s kern2 -n -ex 'target remote 127.0.0.1:7508'
	Leyendo simbolos desde kern2...hecho.
	Remote debugging using 127.0.0.1:7508
	0x0000fff0 in ?? ()
	(gdb) b kmain
	Punto de interrupcion 1 at 0x100129: file kern2.c, line 62.
	(gdb) c
	Continuando.

	Breakpoint 1, kmain (mbi=0x9500) at kern2.c:62
	62	void kmain(const multiboot_info_t *mbi) {
	(gdb) p $esp
	$1 = (void *) 0x104ff4
	(gdb) p/x $eip
	$2 = 0x100129
	(gdb) info frame
	Stack level 0, frame at 0x104ff8:
	eip = 0x100129 in kmain (kern2.c:62); saved eip = 0x10002d
	called by frame at 0x104ffc
	source language c.
	Arglist at 0x104ff0, args: mbi=0x9500
	Locals at 0x104ff0, Previous frame's sp is 0x104ff8
	Saved registers:
	eip at 0x104ff4
\end{lstlisting}
		\end{itemize}
		% EJ: KERN2 CMDLINE
		\subsection{Ej: kern2-cmdline}
		\begin{itemize}
			\item \textit{Mostrar cómo implementar la misma concatenación, de
			manera correcta, usando strncat(3).}
\begin{lstlisting}
	char *strncat(char *dest, const char *src, size_t n) {
		size_t dest_len = strlen(dest);
		size_t i;
		for (i = 0 ; i < n && src[i] != '\0' ; i++) {
			dest[dest_len + i] = src[i];
		}
		dest[dest_len + i] = '\0';
		return dest;
	}
\end{lstlisting}
			\item \textit{Explicar cómo se comporta strlcat(3) si, erróneamente,
			se declarase buf con tamaño 12. ¿Introduce algún error el código?\\}
			No introduce error. Esta función recibe un tercer parámetro que es
			el tamaño máximo que la cadena final tendrá. En este caso, si buf
			tuviera tamaño 12, no seria un problema porque vemos que como tercer
			parámetro recibe el tamaño de la cadena destino, mas el tamaño de la
			cadena fuente mas uno. Entonces como el tamaño máximo es mayor
			tamaño de la cadena destino, se copia la cadena fuente a la destino.
			\item \textit{Compilar el siguiente programa, y explicar por qué se
			imprimen dos líneas distintas, en lugar de la misma dos veces:\\}
\begin{lstlisting}
	#include <stdio.h>
	static void printf_sizeof_buf(char buf[256]) {
		printf("sizeof buf = %zu\n", sizeof buf);
	}
	int main(void) {
		char buf[256];
		printf("sizeof buf = %zu\n", sizeof buf);
		printf_sizeof_buf(buf);
	}
\end{lstlisting}
			Esto es erróneo porque al tener como argumento char buf[256]
			recibe como parámetro un puntero y a este le aplica el operador
			size of, lo cual te da como resultado cuanto ocupa un puntero en
			memoria. Que en general pueden ser 4 u 8 bytes. En cambio, el otro
			caso funciona porque recibe como parámetro el puntero a un buffer y
			lo que hace es calcular las distancia entre este puntero y el
			ultimo caracter. Para que ambos resulten iguales la funcion
			debiera tener como parametro char* buff solamente.
		\end{itemize}
	\section{Concurrencia cooperativa}
		\subsection{Ej: kern2-swap}
		\begin{itemize}
			\item \textit{Explicar, para el stack de cada contador, cuántas
			posiciones se asignan, y qué representa cada una.\\}
			En el caso del segundo stack, se pasan los tres argumentos que la
			función contado\_yield necesita, luego se pasa un puntero a la función
			exit y luego un puntero a la función contador\_yield. Finalmente se
			inicializan las cuatro posiciones siguientes donde se guardaran los
			cuatro registros callee saved. Esto se hace porque cunado se cambia de
			stack, este retira de la stack dichos registros. Por lo tanto
			luego de ello apuntaría a contador\_yield.\\
			Cuando se hace el swap en la función task\_swap y se hace el switch
			entre stacks, la stack actual apunta a la segunda stack. Le saca los
			cuatro registros callee saved y queda apuntando a la función
			contador\_yield. Por lo tanto cuando ejecuta la instrucción ret salta a
			la función contador\_yield. En caso de que el segundo contador termine
			antes, cuando termina el loop principal, salta a otra función de retorno
			que en este caso es la que se guardo antes  y que seria exit.\\
			En el caso del primer contador solo se guardan los tres parámetros. No
			es necesario guardar direcciones de retorno ya que es con este stack que
			se llama a la función, por lo tanto cuando se hace un call a
			contador\_yield se guarda la dirección de retorno en la cuata posición
			del stack.
		\end{itemize}
	\section{Interrupciones: reloj y teclado}
		\subsection{Ej: kern2-idt}
		\begin{itemize}
			\item \textit{¿Cuántos bytes ocupa una entrada en la IDT?\\}
			Una entrada en la IDT tiene un tamaño fijo de 8 bytes.\\
			\item \textit{¿Cuántas entradas como máximo puede albergar la IDT?\\}
			Como máximo la IDT puede albergar 256 entradas, a estas se las conoce como
			descriptores.\\
			\item \textit{¿Cuál es el valor máximo aceptable para el campo limit
			del registro IDTR?\\}
			El valor máximo aceptable para el campo limit depende de la cantidad
			de descriptores válidos
			requeridos para representar las interrupciones y excepciones que
			pueden llegar a ocurrir.
			Debería ser (8*N - 1), expresado en bytes.
			Siendo N la cantidad de entradas de la IDT, el valor 8 refiere 
			a el tamaño fijo de las entradas de la tabla.
			Todos los descriptores vacios deben tener el flag present seteado
			en 0.\\
			\item \textit{Indicar qué valor exacto tomará el campo limit para
			una IDT de 64 descriptores solamente.\\}
			Con una IDT de 64 descriptores el valor del campo limit será
			((8 * 64) -1) = 511\\
			\item \textit{Consultar la sección 6.1 y explicar la diferencia
			entre interrupciones (§6.3) y excepciones (§6.4).\\}
			Las interrupciones y las excepciones son eventos externos al
			procesador son eventos externos
			al procesador, que requieren la atencion del mismo para ser manejadas.

			A primera vista, las diferencias entre interrupciones y excepciones
			radican en el origen y finalidad de las mismas.

			Las interrupciones ocurren en momentos randoms de los programas y
			responden a señales externas al sistema.
			Estas señales son generadas tanto por el Hardware, por ejemplo para
			que el procesador responda al pedido del teclado para detectar el
			ingreso de entrada de datos, o bien por el Software, mediante la
			ejecución de la instrucción INT.

			Las excepciones ocurren cuando el procesador encuentra un problema
			en la ejecución de una instrucción, por ejemplo dividir por cero,
			page faults, violaciones de protección, etc. En las arquitecturas
			\textit{Machine-Check}, tambien es posible que se generen excepciones
			generadas por errores internos del Hardware o errores de Bug,
			llamadas \textit{Machine-Check Exceptions}.\\
		\end{itemize}
		\subsection{Ej: kern2-isr}
			\subsubsection{Sesión de GDB}
			Se debe seguir el mismo guión dos veces:\\
\begin{lstlisting}
.globl breakpoint
breakpoint:
	nop
	test %eax, %eax
	iret
\end{lstlisting}
			\begin{itemize}
				\item \textit{versión A: usando la implementación aumentada
				del manejador:}
% SESION DE GDB VERSION A
\begin{lstlisting}[
columns=fullflexible,
basicstyle=\fontfamily{lmvtt}\selectfont,
]
	0x0000fff0 in ?? ()
	(gdb) display/i $pc
	1: x/i $pc
	=> 0xfff0:	add    %al,(%eax)
	(gdb) b idt_init
	Punto de interrupcion 1 at 0x100586: file interrupts.c, line 27.
	(gdb) c
	Continuando.

	Breakpoint 1, idt_init () at interrupts.c:27
	27	    idt_install(T_BRKPT, breakpoint);
	1: x/i $pc
	=> 0x100586 <idt_init+3>:	push   $0x1000a8
	(gdb) finish
	Correr hasta la salida desde #0  idt_init () at interrupts.c:27
	kmain (mbi=0x9500) at kern2.c:73
	73	    asm("int3");  // (b)
	1: x/i $pc
	=> 0x100193 <kmain+76>:	int3
	(gdb) x/10i $pc
	=> 0x100193 <kmain+76>:	int3
	0x100194 <kmain+77>:	mov    $0x0,%edx
	0x100199 <kmain+82>:	mov    $0xe0,%ecx
	0x10019e <kmain+87>:	mov    $0x12,%eax
	0x1001a3 <kmain+92>:	mov    %edx,%ebx
	0x1001a5 <kmain+94>:	div    %ebx
	0x1001a7 <kmain+96>:	movzbl %cl,%ecx
	0x1001aa <kmain+99>:	movsbl %al,%edx
	0x1001ad <kmain+102>:	mov    $0x100f77,%eax
	0x1001b2 <kmain+107>:	call   0x10009b <vga_write2>
	(gdb) print $esp
	$1 = (void *) 0x104d78
	(gdb) x/xw $esp
	0x104d78:	0x00100f5c
	(gdb) print $cs
	$2 = 8
	(gdb) print $eflags
	$3 = [ ]
	(gdb) print/x $eflags
	$4 = 0x2
	(gdb) stepi
	breakpoint () at idt_entry.S:17
	17	        test %eax, %eax
	1: x/i $pc
	=> 0x1000a9 <breakpoint+1>:	test   %eax,%eax
	(gdb) print $esp
	$5 = (void *) 0x104d6c
\end{lstlisting}
				\item \textit{¿Cuantas posiciones avanzo?}
				Avanzo 12 posiciones
				\item \textit{¿qué representa cada valor?\\}
				Representa el contenido actual de la memoria del stack en
				formato word
\begin{lstlisting}[
columns=fullflexible,
basicstyle=\fontfamily{lmvtt}\selectfont,
]
	(gdb) x/12wx $sp
	0x104d6c:	0x00100194	0x00000008	0x00000002	0x00100f5c
	0x104d7c:	0x00000008	0x00000070	0x00000000	0x00000000
	0x104d8c:	0x00000000	0x00000000	0x00000000	0x00000000
	(gdb) print $eflags
	$6 = [ ]
	(gdb) print/x $eflags
	$7 = 0x2
	(gdb) stepi
	18	        iret
	1: x/i $pc
	=> 0x1000ab <breakpoint+3>:	iret
	(gdb) print $eflags
	$8 = [ PF ZF ]
	(gdb) print/x $eflags
	$9 = 0x46
	(gdb) stepi
	kmain (mbi=0x9500) at kern2.c:78
	78	    asm("div %4"
	1: x/i $pc
	=> 0x100194 <kmain+77>:	mov    $0x0,%edx
	(gdb) x/10i $pc
	=> 0x100194 <kmain+77>:	mov    $0x0,%edx
	0x100199 <kmain+82>:	mov    $0xe0,%ecx
	0x10019e <kmain+87>:	mov    $0x12,%eax
	0x1001a3 <kmain+92>:	mov    %edx,%ebx
	0x1001a5 <kmain+94>:	div    %ebx
	0x1001a7 <kmain+96>:	movzbl %cl,%ecx
	0x1001aa <kmain+99>:	movsbl %al,%edx
	0x1001ad <kmain+102>:	mov    $0x100f77,%eax
	0x1001b2 <kmain+107>:	call   0x10009b <vga_write2>
	0x1001b7 <kmain+112>:	add    $0x10,%esp
	(gdb) print $esp
	$12 = (void *) 0x104d78
	(gdb) x/xw $esp
	0x104d78:	0x00100f5c
	(gdb) print $cs
	$13 = 8
	(gdb) print $eflags
	$10 = [ ]
	(gdb) print/x $eflags
	$11 = 0x2
\end{lstlisting}
				\item \textit{Versión B: con el mismo manejador, pero cambiando
				la instrucción IRET por una instrucción RET.\\}
\begin{lstlisting}[
columns=fullflexible,
basicstyle=\fontfamily{lmvtt}\selectfont,
]
	0x0000fff0 in ?? ()
	(gdb) display/i $pc
	1: x/i $pc
	=> 0xfff0:	add    %al,(%eax)
	(gdb) b idt_init
	Punto de interrupcion 1 at 0x100586: file interrupts.c, line 27.
	(gdb) c
	Continuando.

	Breakpoint 1, idt_init () at interrupts.c:27
	27	    idt_install(T_BRKPT, breakpoint);
	1: x/i $pc
	=> 0x100586 <idt_init+3>:	push   $0x1000a8
	(gdb) finish
	Correr hasta la salida desde #0  idt_init () at interrupts.c:27
	kmain (mbi=0x9500) at kern2.c:73
	73	    asm("int3");  // (b)
	1: x/i $pc
	=> 0x100193 <kmain+76>:	int3
	(gdb) x/10i $pc
	=> 0x100193 <kmain+76>:	int3
	0x100194 <kmain+77>:	mov    $0x0,%edx
	0x100199 <kmain+82>:	mov    $0xe0,%ecx
	0x10019e <kmain+87>:	mov    $0x12,%eax
	0x1001a3 <kmain+92>:	mov    %edx,%ebx
	0x1001a5 <kmain+94>:	div    %ebx
	0x1001a7 <kmain+96>:	movzbl %cl,%ecx
	0x1001aa <kmain+99>:	movsbl %al,%edx
	0x1001ad <kmain+102>:	mov    $0x100f77,%eax
	0x1001b2 <kmain+107>:	call   0x10009b <vga_write2>
	(gdb) print $esp
	$1 = (void *) 0x104d78
	(gdb) x/xw $esp
	0x104d78:	0x00100f5c
	(gdb) print $cs
	$2 = 8
	(gdb) print $eflags
	$3 = [ ]
	(gdb) print/x $eflags
	$4 = 0x2
	(gdb) stepi
	breakpoint () at idt_entry.S:26
	26	        test %eax, %eax
	1: x/i $pc
	=> 0x1000a9 <breakpoint+1>:	test   %eax,%eax
	(gdb) print $esp
	$5 = (void *) 0x104d6c
\end{lstlisting}
				\item \textit{¿Cuantas posiciones avanzo?}
				Avanzo 12 posiciones
				\item \textit{¿qué representa cada valor?\\}
				Representa el contenido actual de la memoria del stack en
				formato word
\begin{lstlisting}[
columns=fullflexible,
basicstyle=\fontfamily{lmvtt}\selectfont,
]
	(gdb) x/12wx $sp
	0x104d6c:	0x00100194	0x00000008	0x00000002	0x00100f5c
	0x104d7c:	0x00000008	0x00000070	0x00000000	0x00000000
	0x104d8c:	0x00000000	0x00000000	0x00000000	0x00000000
	(gdb) print $eflags
	$6 = [ ]
	(gdb) print/x $eflags
	$7 = 0x2
	(gdb) stepi
	27	        ret
	1: x/i $pc
	=> 0x1000ab <breakpoint+3>:	ret
	(gdb) print $eflags
	$8 = [ PF ZF ]
	(gdb) print/x $eflags
	$9 = 0x46
	(gdb) stepi
	kmain (mbi=0x9500) at kern2.c:78
	78	    asm("div %4"
	1: x/i $pc
	=> 0x100194 <kmain+77>:	mov    $0x0,%edx
	(gdb) x/10i $pc
	=> 0x100194 <kmain+77>:	mov    $0x0,%edx
	0x100199 <kmain+82>:	mov    $0xe0,%ecx
	0x10019e <kmain+87>:	mov    $0x12,%eax
	0x1001a3 <kmain+92>:	mov    %edx,%ebx
	0x1001a5 <kmain+94>:	div    %ebx
	0x1001a7 <kmain+96>:	movzbl %cl,%ecx
	0x1001aa <kmain+99>:	movsbl %al,%edx
	0x1001ad <kmain+102>:	mov    $0x100f77,%eax
	0x1001b2 <kmain+107>:	call   0x10009b <vga_write2>
	0x1001b7 <kmain+112>:	add    $0x10,%esp
	(gdb) print $esp
	$10 = (void *) 0x104d70
	(gdb) x/xw $esp
	0x104d70:	0x00000008
	(gdb) print $cs
	$11 = 8
	(gdb) print $eflags
	$12 = [ PF ZF ]
	(gdb) print/x $eflags
	$13 = 0x46
\end{lstlisting}
\item \textit{Para cada una de las siguientes maneras de
				guardar/restaurar registros en breakpoint, indicar si es
				correcto (en el sentido de hacer su ejecución “invisible”), y
				justificar por qué:}
				\begin{itemize}
					\item Opcion A
\begin{lstlisting}
	breakpoint:
	pusha
	...
	call vga_write2
	popa
	iret
\end{lstlisting}
					\item Opcion B
\begin{lstlisting}
	breakpoint:
	push %eax
	push %edx
	push %ecx
	...
	call vga_write2
	pop %ecx
	pop %edx
	pop %eax
	iret
\end{lstlisting}
					\item Opcion C
\begin{lstlisting}
	breakpoint:
	push %ebx
	push %esi
	push %edi
	...
	call vga_write2
	pop %edi
	pop %esi
	pop %ebx
	iret
\end{lstlisting}
\end{itemize}
					La opcion correcta es la B, ya que al llamar a la funcion,
					se supone que los registros callee saved, son salvados de
					manera tal que al volver de la funcion, estos mantienen
					su valor. En cambio, los registros caller saved no. Por lo
					tanto, la opcion A hace trabajo de mas al salvar todos los
					registros y la C no funcionaria ya que salva los registros
					callee saved que no es necesario ya que de eso se encarga la
					funcion llamada pero no salva los caller saved y al volver
					se podria perder si valor.\\
			\end{itemize}
		\subsection{Ej: kern2-div}
		\begin{itemize}
			Explicar el funcionamiento exacto de la línea asm(\ldots):\\
\begin{lstlisting}
	void kmain(const multiboot_info_t *mbi) {
	int8_t linea;
	uint8_t color;

	// ...

	idt_init();
	irq_init();

	asm("div %4"
	: "=a"(linea), "=c"(color)
	: "0"(18), "1"(0xE0), "b"(1), "d"(0));

	vga_write2("Funciona vga_write2?", linea, color);
	}

\end{lstlisting}
			\item \textit{¿qué cómputo se está realizando?\\}
			Esta dividiendo \%eax por \%ebx que le asigna el valor uno\\
			\item \textit{¿de dónde sale el valor de la variable color?\\}
			El valor se lo asigna en caso de que la division haya sido exitosa
			y el valor sale del primer registro "1"(0xE0) por eso.\\
			\item \textit{¿por qué se da valor 0 a \%edx?\\}
			Se le da valor cero porque el resto de la division se produce entre
			EDX:EAX y el EBX.\\
		\end{itemize}
\end{document}